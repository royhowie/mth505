\documentclass{article}
\usepackage{amsfonts}
\usepackage{amsmath}
\usepackage{amssymb}
\usepackage{amsthm}
\usepackage{centernot}
\usepackage{enumitem}
\usepackage{parskip}
\usepackage{titling}

\newcommand{\Z}{\mathbb{Z}}

\setcounter{section}{4}

\begin{document}

\title{\vspace{-3cm}MTH 505 Homework 4}
\author{Roy Howie}
\date{April 14, 2017}
\maketitle

% problem 4.1
\subsection{Squares Mod 4}
  \begin{enumerate}[label=\textbf{(\alph*)}]
    \item{
      Note $0^2\equiv0$, $1^2\equiv1$, $2^2\equiv0$, and $3^2\equiv1$ modulo 4.
      Thus 0 and 1 are the only square elements of $\Z_4$.
    }
    \item{
      First, for $n\in\Z$, note $n\equiv1\pmod{2}$ implies $n^2\equiv1\pmod{4}$.
      Next, let $x,y,z\in\Z$ such that $x^2+y^2=z^2$. Suppose $x$ and $y$ are
      both odd, then $x^2\equiv y^2\equiv1\pmod{4}$, implying $z^2\equiv2
      \pmod{4}$. A contradiction, as 2 is not a square elements of $\Z_4$.
      Hence, $x$ and $y$ cannot both be odd.
      \qed
    }
  \end{enumerate}

% problem 4.2
\subsection{Fermat's Last Theorem}
  Consider (FLT): $x^4+y^4=z^2$ for $x,y,z\in\Z$ with $xyz\ne0$.
  \begin{enumerate}[label=\textbf{(\alph*)}]
    \item{
      Suppose FLT has a solution $(x_1,y_1,z_1)\in\Z^3$, then there is another
      solution $(x_n,y_n,z_n)\in\Z^3$ with $\gcd(x_n,y_n)=1$ and $0<z_n<|z|$.

      Let $p$ be a prime divisor of both $x_1$ and $y_1$. If no such number
      exists, then $\gcd(x_1,y_1)=1$ and we are done. Otherwise, $(x_1/p,y_1/p,
      z_1/p^2)$ is another solution, as $(x_1/p)^4+(y_1/p)^4=(x_1^4+y_1^4)/p^4=
      z_1^2/p^4$. Recurse.
    }
    \item{
      Suppose $(x,y,z)$ is a solution to FLT with $\gcd(x,y)=1$.

      Note $x$ and $y$ cannot both be odd, as $x^4+y^4=(x^2)^2+(y^2)^2=z^2$.
      WLOG, assume $x$ is odd, $y$ is even, and $z=|z|$. There are then coprime
      $u,v\in\Z$ with $v>0$ such that $x^2=v^2-u^2$, $y=2uv$, and $z=u^2+v^2$.
      As $u^2+x^2=v^2$, there are again coprime $r,s\in\Z$ with
      $s>0$ such that $x=s^2-r^2$, $u=2rs$, and $v=r^2+s^2$.

      Next, consider $y^2=2uv$. Since $y$ is even, 4 divides $y^2$. Note $u$
      is even, as $x$ is odd and $u^2+x^2=v^2$. Thus, $(y/2)^2=(u/2)v$. Recall
      $\gcd(u,v)=1$ and that if a prime $p$ divides a number $t^2$, then $p^2$
      does too. Therefore, if $p^2$ divides $(y/2)^2$, then $p^2$ divides either
      $u/2$ or $v$. Hence, $u/2$ and $v$ are perfect squares each. By similar
      argument, as $u=2rs$ and $u/2$ is an even square number, $r$ and $s$ are
      also perfect squares.

      Therefore, there are $(m,n,l)\in\Z^3$ with $s=m^2$, $r=n^2$, and $v=l^2$.
      Note $v=r^2+s^2$, so $l^2=m^4+n^4$ and $(m,n,l)$ is another solution to
      FLT. Furthermore, $\gcd(m,n)=1$, as$r$ and $s$ were coprime; $mnl\ne0$, as
      $rsv\ne0$, and $0<l<|z|$, as $z=u^2+v^2$ with $v$ positive and $v=l^2$.

      We are back to square one, contradicting the well ordering of the
      integers, as every solution $(a,b,c)$ to FLT produces another solution
      $(d,e,f)$ with $0<f<|c|$. Hence, no solution to FLT exists.
      \qed
    }
  \end{enumerate}

% problem 4.3
\subsection{Rational Points on a Hyperbola}

% problem 4.4
\subsection{A Hyperbola with No Rational Points}

\end{document}
