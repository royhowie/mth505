\documentclass{article}
% \usepackage[margin=1in]{geometry}
\usepackage{amsfonts}
\usepackage{amsmath}
\usepackage{amssymb}
\usepackage{amsthm}
\usepackage{centernot}
\usepackage{enumitem}
\usepackage{parskip}
\usepackage{titling}

\newcommand{\Z}{\mathbb{Z}}

\setcounter{section}{3}

\begin{document}

\title{\vspace{-3cm}MTH 505 Homework 3}
\author{Roy Howie}
% \date{March 10, 2017}
\maketitle

% problem 3.1
\subsection{Infinitely Many Primes}
  Suppose there is a finite number of positive primes $p_1,p_2,\cdots,p_k$ and
  consider the number $N=1+\prod_{i=1}^k{p_i}$. Note that $N\ge2$, so it must
  have a positive prime factor $q$. Furthermore, for all $i$, one has $N\equiv1
  \pmod{p_i}$. But $q$ divides $N$, so there is no $p_i$ equal to $q$. This is a
  contradiction, as we assumed our list contained all positive prime numbers.
  Hence, no such finite list exists.
  \qed

% problem 3.2
\subsection{Infinitely Many Primes $\equiv$ 3 Modulo 4}
  Let $n\equiv3\pmod{4}$ be a positive integer, then it can be written as the
  product of positive primes: $p_1p_2\cdots p_k$. Note that $p_i$ is either
  equal to 1 or 3 modulo 4. Assume $p_i\equiv1\pmod{4}$ for all $i$. This is a
  contradiction, as then $n\equiv p_1p_2\cdots p_k\equiv1*1*\cdots
  *1\equiv1\pmod{4}$. But we assumed $n\equiv3\pmod{4}$. Hence, $n$ has at least
  one prime factor which is 3 modulo 4.

  Next, suppose there is a finite number of primes congruent to 3 modulo 4: $3<
  p_1 <p_2<\cdots<p_k$. Consider $N=3+4\prod_{i=1}^k{p_i}$. Note that $N\equiv3
  \pmod{4}$, so it has a prime factor $q\equiv3\pmod{4}$. Furthermore, note
  that, for all $i$, one has $N\equiv3\pmod{p_i}$. But $q$ divides $N$, so there
  is no $p_i$ equal to $q$. This is a contradiction, as our list purportedly
  contained all primes congruent to 3 modulo 4. Therefore, no such finite list
  exists.
  \qed

% problem 3.3
\subsection{Infinitely Many Primes $\equiv$ 1 Modulo 4}
  Suppose there is a finite number of primes congruent to 1 modulo 4: $p_1,p_2,
  \cdots,p_k$. Let $x=2p_1p_2\cdots p_k$ and consider $N=1+x^2$. Note that $N
  \equiv1\pmod{4}$, as $4\mid x^2$. Furthermore, for all $i$, one has $p_i\mid
  x$, so $N\equiv1\pmod{p_i}$.

  If $N$ is prime, this is a contradiction, for then there is a prime number
  congruent to 1 modulo 4 not found in the above list.

  Otherwise, $N$ has a prime divisor $q$. Let $\varphi$ be the totient function
  and note that $\varphi(q)=q-1$. But then $x^2\equiv N-1\equiv-1\pmod{q}$, so
  $x^4\equiv1\pmod{q}$. This implies $4\mid (q-1)$, or that $q\equiv1\pmod{4}$.
  Recall that none of the primes in our list divided $N$. However, $q\mid N$, so
  $q$ is a prime congruent to 1 modulo 4 not present in our list of primes. This
  is a contradiction, as our list supposedly contained all primes congruent to 1
  modulo 4.

  Hence, no such list exists.
  \qed

% problem 3.4
\subsection{A Useful Lemma}
  Let $a,b,c\in\Z$ with $\gcd(a,b)=1$. We wish to show $a\mid c$ and $b\mid c$
  implies $ab\mid c$.

  Note that there exist $d,e,x,y\in\Z$ such that $ad=c=be$ and $ax+by=1$. Thus,
  \begin{equation*}
    c = cax+cby=(be)ax+(ad)by=ab(ex+dy)
  \end{equation*}
  So $ab$ divides $c$.
  \qed

\end{document}
