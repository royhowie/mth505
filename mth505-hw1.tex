\documentclass{article}
\usepackage[margin=1in]{geometry}
\usepackage{amsfonts}
\usepackage{amsmath}
\usepackage{amsthm}
\usepackage{parskip}
\usepackage{enumitem}

\newcommand{\N}{\mathbb{N}}
\newcommand{\Z}{\mathbb{Z}}
\newcommand{\Q}{\mathbb{Q}}

\newtheorem*{lemma}{Lemma}

\begin{document}

\title{MTH 505 Homework 1}
\author{Roy Howie}
\date{February 1, 2017}
\maketitle

\section{Developing $\N$}
  Completed in the following order: a, b, c, d, e, h, f, g.
  \begin{enumerate}[label=\textbf{(\alph*)}]
    \item{
      Let $S=\{n\in\N\mid n=0 \ \vee \ \exists m\in\N\text{ such that }\sigma(m)
      =n\}$. Note $0\in S$. Next, suppose $x\in S$, then $\sigma(x)\in S$, as
      there exists $m \in \N$ such that $\sigma(m)=\sigma(x)$: namely, $x$
      itself. Hence, $\bf{P4}\Rightarrow S=\N$.

      Next, fix $n\in\N$ and suppose $\exists m_1,m_2\in\N$ such that
      $\sigma(m_1)=n=\sigma(m_2)$ and $m\ne n$. This is a contradiction. By
      $\bf{P3}$, $\sigma(m_1)=\sigma(m_2)$ implies $m_1=m_2$. Hence,
      predecessors are unique.
      \hfill $\square$
    }
    \item{
      Fix $a,b\in\N$ and let $S=\{c\in\N\mid a+(b+c)=(a+b)+c\}$. By $\bf{(1)}$,
      $0\in S$ as $a+(b+0)=a+(b)=(a+b)+0$. Next, suppose $n \in S$, then
      \begin{align*}
        a+(b+\sigma(n)) &= a+\sigma(b+n)    \tag{2} \\
                        &= \sigma(a+(b+n))  \tag{2} \\
                        &= \sigma((a+b)+n)  \tag{IH}\\
                        &= (a+b)+\sigma(n)  \tag{2}
      \end{align*}
      Hence, $\sigma(n) \in S$, so $\bf{P4}\Rightarrow S = \N$.
      \hfill $\square$
    }
    \item{
      Let $S=\{n\in\N\mid0+n=n\}$. By $\bf{(1)}$, $0\in S$ as $0+0=0$. Next,
      suppose $x\in S$, then
      \begin{align*}
        0 + \sigma(x) &= \sigma(0+x)  \tag{2} \\
                      &= \sigma(x)    \tag{IH}
      \end{align*}
      Hence, $\sigma(x) \in S$, so $\bf{P4}\Rightarrow S=\N$. In addition, via
      $\bf{(1)}$, one has $0+n=n+0$ for all $n\in\N$.

      Let $R=\{n\in\N\mid n+\sigma(0)=\sigma(0)+n\}$. By above, $0\in R$, as
      $0+\sigma(0)=\sigma(0)+0$. Suppose $x\in R$, then
      \begin{align*}
        \sigma(x)+\sigma(0) &= \sigma(\sigma(x)+0)    \tag{2} \\
                            &= \sigma(\sigma(x))      \tag{1} \\
                            &= \sigma(\sigma(x+0))    \tag{1} \\
                            &= \sigma(x+\sigma(0))    \tag{2} \\
                            &= \sigma(\sigma(0)+x)    \tag{IH} \\
                            &= \sigma(0)+\sigma(x)    \tag{2}
      \end{align*}
      Hence, $\sigma(x) \in R$, so $\bf{P4}\Rightarrow R=\N$.
      \hfill $\square$
    }
    \item{
      Fix $a\in\N$ and let $S=\{b\in\N\mid a+b=b+a\}$. By $\bf{(c)}$, $0\in S$.
      Suppose $x\in S$, then
      \begin{align*}
        a+\sigma(x) &= \sigma(a+x)      \tag{2} \\
                    &= \sigma(x+a)      \tag{IH}\\
                    &= \sigma((x+a)+0)  \tag{1} \\
                    &= (x+a)+\sigma(0)  \tag{2} \\
                    &= x+(a+\sigma(0))  \tag{b} \\
                    &= x+(\sigma(0)+a)  \tag{c} \\
                    &= (x+\sigma(0))+a  \tag{b} \\
                    &= \sigma(x+0)+a    \tag{2} \\
                    &= \sigma(x)+a      \tag{1}
      \end{align*}
      Hence, $\sigma(x) \in S$, so $\bf{P4}\Rightarrow S=\N$.
      \hfill $\square$
    }
    \item{
      Fix $b,c\in\N$ and let $S=\{a\in\N\mid(b+c)a=ba+ca\}$. By $\bf{(1)}$ and
      $\bf{(3)}$, $0\in S$, as $(b+c)*0=0+0$. Suppose $x\in S$, then
      \begin{align*}
        (b+c)\sigma(x)  &= (b+c)x+(b+c)           \tag{4}  \\
                        &= (bx+cx)+(b+c)          \tag{IH} \\
                        &= (bx+b)+(cx+c)          \tag{b,d}\\
                        &= b\sigma(x)+c\sigma(x)  \tag{4}
      \end{align*}
      Hence, $\sigma(x) \in S$, so $\bf{P4}\Rightarrow S=\N$.
      \hfill $\square$
    }
    \item{
      Fix $a,b\in\N$ and let $S=\{c\in\N\mid(ab)c=a(bc)\}$. By $\bf{(3)}$, $0\in
      S$, as $a(b*0)=0=(ab)*0$. Suppose $x\in S$, then
      \begin{align*}
        a(b\sigma(x)) &= a(bx+b)            \tag{4}  \\
                      &= a(bx)+ab           \tag{h,e}\\
                      &= (ab)x+ab           \tag{IH} \\
                      &= (ab)x+ab\sigma(0)  \tag{h}  \\
                      &= (ab)(x+\sigma(0))  \tag{e}  \\
                      &= (ab)\sigma(x+0)    \tag{2}  \\
                      &= (ab)\sigma(x)      \tag{1}
      \end{align*}
      Hence, $\sigma(x)\in S$, so $\bf{P4}\Rightarrow S=\N$.
      \hfill $\square$
    }
    \item{
      By $\bf{(h)}$ and $\bf{(4)}$, $\sigma(a)b=b\sigma(a)=ba+b=ab+b$.
      \hfill $\square$
    }
    \item{
      Let $S=\{n\in\N\mid0*a=0\}$. By $\bf{(3)}$, $0\in S$, as $0*0=0$. Suppose
      $x\in S$, then
      \begin{align*}
        0*\sigma(x) &= 0*x+0    \tag{4} \\
                    &= 0+0      \tag{IH}\\
                    &= 0        \tag{1}
      \end{align*}
      Hence, $\sigma(x)\in S$, so $\bf{P4}\Rightarrow S=\N$.

      Let $R=\{a\in\N\mid\sigma(0)*a=a\}$. By $\bf{(1)}$, $0\in S$, as
      $\sigma(0)*0=0$. Suppose $x\in S$, then
      \begin{align*}
        \sigma(0)\sigma(x)  &= \sigma(0)x+\sigma(0) \tag{4}\\
                            &= x+\sigma(0)          \tag{IH}\\
                            &= \sigma(x+0)          \tag{2} \\
                            &= \sigma(x)            \tag{1}
      \end{align*}
      Hence, $\sigma(x) \in R$, so $\bf{P4}\Rightarrow R=\N$.

      Fix $a\in\N$ and let $Q=\{b\in\N\mid ab=ba\}$. By above, $0\in Q$ because
      $0*a=a*0$. Suppose $x\in Q$, then
      \begin{align*}
        a\sigma(x)  &= ax+a           \tag{4}     \\
                    &= xa+a           \tag{IH}    \\
                    &= xa+\sigma(0)a  \tag{above} \\
                    &= (x+\sigma(0))a \tag{e}     \\
                    &= \sigma(x+0)a   \tag{2}     \\
                    &= \sigma(x)a     \tag{1}
      \end{align*}
      Hence, $\sigma(x) \in Q$, so $\bf{P4}\Rightarrow Q=\N$.
      \hfill $\square$
    }
  \end{enumerate}

\hfill\break
\begin{lemma}{(Additive Cancellation)}
  For $a,b,c \in \N$, if $a+c=b+c$, then $a=b$.
\end{lemma}
\begin{proof}
  Fix $a,b\in\N$ and let $S=\{c\in\N\mid a+c=b+c\Rightarrow a=b\}$. Note $0\in
  S$. Suppose $x\in S$, then
  \begin{align*}
    a+\sigma(x) &= b+\sigma(x)          \\
    \sigma(a+x) &= \sigma(b+x) \tag{2}  \\
            a+x &= b+x         \tag{P3} \\
              a &= b           \tag{IH}
  \end{align*}
  Hence $\sigma(x) \in S$, so $\bf{P4}\Rightarrow S=\N$.
\end{proof}

\section{From $\N$ to $\Z$}
  \begin{enumerate}[label=\textbf{(\alph*)}]
    \item{
      To show $[a,b] \sim [c,d] \iff a+d=c+b$ is an equivalence relation:
      \begin{enumerate}[label=(\arabic*)]
        \item{
          $[a,b] \sim [a,b] \Rightarrow a+b=a+b$
        }
        \item{
          $[a,b] \sim [c,d] \Leftrightarrow [c,d] \sim [a,b]$. Note that
          $[a,b]\sim[c,d]
            \Leftrightarrow a+d=c+b
            \Leftrightarrow c+b=a+d
            \Leftrightarrow [c,d]\sim[a,b]
          $.
        }
        \item{
          $[a,b]\sim[c,d]\wedge[c,d]\sim[e,f]\Rightarrow[a,b]\sim[e,f]$. By
          assumption, $a+d=b+c$ and $c+f=e+d$.
          \begin{align*}
                a+d &= c+b      \\
            (a+d)+f &=(c+b)+f   \\
                    &= b+(c+f)  \\
                    &= b+(e+d)  \\
                    &= (e+b)+d  \\
            (a+f)+d &=(e+b)+d
          \end{align*}
          Hence, by the Additive Cancellation Lemma, $a+f=e+b$, so
          $[a,b]\sim[e,f]$.
          \hfill $\square$
        }
      \end{enumerate}
    }
    \item{
      \begin{enumerate}[label=(\arabic*)]
        \item{
          To show $[a,b]+[c,d]=[a+c,b+d]$ is well-defined, consider
          $[a,b]\sim[a',b']$ and $[c,d]\sim[c',d']$. Note that $a+b'=a'+b$ and
          $c+d'=c'+d$. Therefore, $(a+b')+(c+d')=(a'+b)+(c'+d)$. Some shuffling
          of terms yields $(a+c)+(b'+d')=(a'+c')+(b+d)$, implying $[a+c,b+d]\sim
          [a'+c',b'+d']$, so addition on equivalence classes is indeed
          well-defined.
        }
        \item{
          To show $[a,b]*[c,d]=[ac+bd,ad+bc]$ is well-defined, again consider
          $[a,b]\sim[a',b']$ and $[c,d]\sim[c',d']$.

          We want $[ac+bd,ad+bc]\sim[a'c'+b'd',a'd'+b'c']$, or
          $(ac+bd)+(a'd'+b'c')=(a'c'+b'd')+(ad+bc)$. We have $a+b'=a'+b$ and
          $c+d'=c'+d$. We only need imagination:
          \begin{align*}
            (a+b')+(a'+b)+(c+d')+(c'+d)
                              &= (a'+b)+(a+b')+(c'+d)+(c+d') \\
            (a+b')c+(a'+b)d+(c+d')a'+(c'+d)b'
                              &= (a'+b)c+(a+b')d+(c'+d)a'+(c+d')b' \\
            (ac+b'c)+(a'd+bd)+(ca'+d'a')+(c'b'+db')
                              &= (a'c+bc)+(ad+b'd)+(c'a'+da')+ (cb'+d'b') \\
            (ac+bd)+(d'a'+c'b')+(b'c+a'd+ca'+db')
                              &= (c'a'+d'b')+(ad+bc)+(a'c+b'd+da'+cb') \\
            (ac+bd)+(d'a'+c'b')
                              &= (c'a'+d'b')+(ad+bc)
          \end{align*}
          The general rule is to keep numbers of the form $ab$ or $a'b'$ but not
          $a'b$ or $ab'$. Nevertheless, the last line implies the desired
          result, so multiplication on equivalence classes is well-defined.
          \hfill $\square$
        }
      \end{enumerate}
    }
    \item{
      \begin{enumerate}[label=(\arabic*)]
        \item{
          Associativity of $+:\Z\to\Z$
          \begin{align*}
            ([a,b]+[c,d])+[e,f] &= [a+c,b+d]+[e,f]  \\
                                &= [(a+c)+e,(b+d)+f]\\
                                &= [a+(c+e),b+(d+f)]\\
                                &= [a,b]+[c+e,d+f]  \\
                                &= [a,b]+([c,d]+[e,f])
          \end{align*}
        }
        \item{
          Commutativity of $+:\Z\to\Z$
          \begin{align*}
            [a,b]+[c,d] &= [a+c,b+d]\\
                        &=[c+a,d+b] \\
                        &=[c,d]+[a,b]
          \end{align*}
        }
        \item{
          Associativity of $*:\Z\to\Z$
          \begin{align*}
            ([a,b]*[c,d])*[e,f] &= [ac+bd,ad+bc]*[e,f]                  \\
                                &= [(ac+bd)e+(ad+bc)f,(ac+bd)f+(ad+bc)e]\\
                                &= [ace+bde+adf+bcf,acf+bdf+ade+bce]    \\
                                &= [a(ce+df)+b(cf+de),a(cf+de)+b(df+ce)]\\
                                &= [a,b]*[ce+df,cf+de]                  \\
                                &= [a,b]*([c,d]*[e,f])
          \end{align*}
        }
        \item{
          Commutativity of $*:\Z\to\Z$
          \begin{align*}
            [a,b]*[c,d] &= [ac+bd,ad+bc] \\
                        &= [ca+db,cb+da] \\
                        &= [c,d]*[a,b]
          \end{align*}
        }
        \hfill $\square$
      \end{enumerate}
    }
    \item{
      First, additive inverses are added, which turns $\N$ into an abelian group
      $\Z'$. Next, a second binary operation, multiplication, is added to
      transform the abelian group $\Z'$ into the commutative ring $\Z$. Every
      $n\in\N$ has an ``additive inverse'' in the form of the members of the
      equivalence class of $[0,n]$.
    }
  \end{enumerate}

\vfill
\section{From $\Z$ to $\Q$}
  \begin{enumerate}[label=\textbf{(\alph*)}]
    \item{
      To show $[a,b]\sim[c,d] \Leftrightarrow ad=bc$ is an equivalence relation:
      \begin{enumerate}[label=(\arabic*)]
        \item{
          $[a,b]\sim[a,b] \Rightarrow ab=ba$
        }
        \item{
          $[a,b]\sim[c,d]\Leftrightarrow[c,d]\sim[a,b]$ Note that
          $[a,b]\sim[c,d]
            \Rightarrow ad = bc
            \Rightarrow cb = da
            \Rightarrow [c,d]\sim[a,b]
          $.
        }
        \item{
          $[a,b]\sim[c,d]\wedge[c,d]\sim[e,f]\Rightarrow[a,b]\sim[e,f]$. By
          assumption, $ad=bc$ and $cf=de$.
          \begin{align*}
            (ad)f &= (bc)f \\
                  &= b(cf) \\
                  &= b(de) \\
            (af)d &= (be)d
          \end{align*}
          Hence, by multiplicative cancellation, $af=be$, so $[a,b]\sim[e,f]$.
          \hfill $\square$
        }
      \end{enumerate}
    }
    \item{
      \begin{enumerate}[label=(\arabic*)]
        \item{
          To show $[a,b]*[c,d]=[ac,bd]$ is well-defined, consider $[a,b]\sim
          [a',b']$ and $[c,d]\sim[c',d']$. Note that $ab'=ba'$ and $cd'=dc'$.
          Therefore, $(ab')(cd')=(ba')(dc')$. Shuffling terms gives $(ac)(b'd')=
          (bd)(a'c')$, implying $[ac,bd]\sim[a'c',b'd']$. Hence, multiplication
          is well-defined.
        }
        \item{
          To show $[a,b]+[c,d]=[ad+bc,bd]$ is well-defined, again consider
          $[a,b]\sim[a',b']$ and $[c,d]\sim[c',d']$.
          \begin{align*}
            (ab')+(cd')         &= (ba')+(dc')\\
            (bb'+dd')(ab'+cd')  &= (ba'+dc')(dd'+bb')\\
            (ab'dd'+cd'bb')+(ab'bb'+cd'dd')
                                &= (a'bdd'+c'dbb')+(ba'bb'+dc'dd')\\
            ab'dd'+cd'bb'       &= a'bdd'+c'dbb'\\
            (ad)(b'd')+(bc)(b'd')
                                &= (a'd')(bd)+(b'c')(bd)\\
            (ad+bc)(b'd')       &= (bd)(a'd'+b'c')
          \end{align*}
          Hence, $[ad+bc,bd]\sim[a'd'+b'c',b'd']$ and addition is well-defined.
        }
        \hfill $\square$
      \end{enumerate}
    }
    \item{
      Too tedious.
    }
    \item{
      $\Z$ is a commutative ring, whereas $\Q$ is a field. Thus, to transition
      from $\Z$ to $\Q$, one needs multiplicative inverses for all nonzero
      integers. Thus, every $q\in\Q$ is represented by an equivalence class
      $[a,b]$ and has a corresponding ``multiplicative inverse'' represented by
      the equivalence class $[b,a]$.
    }
  \end{enumerate}

\end{document}
