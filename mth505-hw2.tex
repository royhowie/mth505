\documentclass{article}
\usepackage[margin=1in]{geometry}
\usepackage{amsfonts}
\usepackage{amsmath}
\usepackage{amssymb}
\usepackage{amsthm}
\usepackage{braket}
\usepackage{centernot}
\usepackage{enumitem}
\usepackage{mathtools}
\usepackage{parskip}
\usepackage{titlesec}

\newcommand{\N}{\mathbb{N}}
\newcommand{\Z}{\mathbb{Z}}

\DeclarePairedDelimiter\floor{\lfloor}{\rfloor}

\setcounter{section}{2}

\titleformat{\subsection}[runin]
  {\normalfont\large\bfseries}{\thesubsection}{1em}{}

\begin{document}

\title{\vspace{-2.05cm}MTH 505 Homework 2}
\author{Roy Howie}
\date{February 20, 2017}
\maketitle

Fix $a,b\in\N$ with $ab\ne0$ and let $R_{a,b,c}=\set{(x,y)\in\N^2\mid ax+by=c}$.

\subsection{} % 2.1
  Consider $a,b,c\in\N$ with $d=\gcd(a,b)$, where $a=da'$ and $b=db'$.
  \begin{enumerate}[label=\textbf{(\alph*)}]
    \item{
      Let $d\centernot\mid c$, then there exists $q,r\in\Z$ such that $c=qd+r$
      and $0<r<d$. Note that $ax+by=c=qd+r$, or that $d(a'x+b'y-q)=r$. This
      implies $d\mid r$, a contradiction, as $0<r<d$. Hence, $R_{a,b,c}=
      \varnothing$.
    }
    \item{
      Let $d\mid c$ with $c=dc'$ and let $R_{a',b',c'}=\set{(x,y)\in\N^2\mid
      a'x+b'y=c'}$.

      Suppose $(x,y)\in R_{a,b,c}$, then $ax+by=c$. This implies
      $d(a'x+b'y)=dc'$, or that $a'x+b'y=c'$, meaning $(x,y)\in R_{a',b',c'}$.
      Hence, $R_{a,b,c}\subset R_{a',b',c'}$.

      Conversely, suppose $(x',y')\in R_{a',b',c'}$, then $a'x'+b'y'=c'$,
      implying $d(a'x'+b'y')=dc'$. Therefore, $ax'+by'=c$, so
      $(x',y')\in R_{a,b,c}$. Hence, $R_{a',b',c'}\subset R_{a,b,c}$

      Thus, $R_{a,b,c}=R_{a',b',c'}$.
    }
    \qed
  \end{enumerate}

\subsection{} % 2.2
  Let $a,b,c\in\N$ with $ab\ne0$ and $\gcd(a,b)=1$. If $R_{a,b,c}$ is nonempty,
  prove there exists unique $(u,v)\in R_{a,b,c}$ such that $0\leq u<b$.

  \textit{Existence.}
  Let $(x,y)\in R_{a,b,c,}$, then there exists unique $q,r\in\N$ such that
  $x=qb+r$ and $0\leq r<b$. If $q=0$, then $x<b$, so $(u,v)=(x,y)$. Otherwise,
  $ax+by=c$ implies $a(qb+r)+by=c$, or that $ar+b(y+aq)=c$, meaning
  $(u,v)=(r,y+aq)$.

  \textit{Uniqueness.}
  Suppose $(x,y)$ can be reduced to two pairs $(u,v)$ and $(\tilde{u},
  \tilde{v})$ with $0\leq u,\tilde{u}<b$. Then $au+bv=c$ and $a\tilde{u}+b
  \tilde{v}=c$, implying $au\equiv c\pmod{b}$ and $a\tilde{u}\equiv c\pmod{b}$.
  But $a$ and $b$ are coprime, so $u\equiv c\equiv\tilde{u}\pmod{b}$.
  Furthermore, $0\leq u,\tilde{u}<b$, so $u$ and $\tilde{u}$ inhabit the same
  residue class and are therefore equal.
  \qed

\subsection{} % 2.3
  Let $a,b\in\N$ be coprime with $ab\ne0$. If $c=ab-a-b$, prove $R_{a,b,c}=
  \varnothing$.

  Note that, in \textbf{2.2}, we have existence iff $R_{a,b,c}$ is nonempty.
  Uniqueness, however, is a result of the Division Algorithm. That is, there is
  only one pair $(u,v)\in\Z^2$ satisfying $au+bv=ab-a-b$ with $0\leq u<b$.

  Next, note that $(u,v)=(b-1,-1)$ is a solution to $au+bv=ab-a-b$. Thus, if
  $R_{a,b,c}$ is nonempty, then it contains a pair $(x,y)$ without a
  corresponding $(u,v)$ satisfying the conditions laid out in \textbf{2.2}. But
  for all $(x,y)$ in $R_{a,b,c}$, $(u,v)$ existed and was unique! Hence,
  $R_{a,b,c}$ is empty.
  \qed

\subsection{} % 2.4
  Let $a,b\in\N$ be coprime with $ab\ne0$. Prove every number $c>ab-a-b$ is
  $(a,b)$-representable.
  \begin{enumerate}[label=\textbf{(\alph*)}]
    \item{
      Suppose $a=1$ or $b=1$. Without loss of generality, let $a=1$. We wish to
      show every number $c>1*b-1-b=-1$ is $(a,b)$-representable. This is clearly
      the case, as, for all $n\in\N$, $n=a*n$.
    }
    \item{
      Assume $a,b\geq2$.

      As $\gcd(a,b)=1$, there exist $x',y'\in\Z$ such that
      $ax'+by'=1$. Furthermore, as in \textbf{2.2}, we may further constrain
      $x'$ to $0\leq x'<b$. Next, note that $x'>0$, as $x=0$ implies $b=1$, but
      we assumed $b\geq2$. Hence, $(x'-1)\in\N$.

      Likewise, assume $(y'+a-1)<0$. This implies $y'\leq -a$. But then $ax'+by'
      =1\leq ax'+b(-a)=a(x'-b)$, meaning $x'>b$. This is a contradiction, so
      $(y'+a-1)\in\N$.

      Therefore, $(1-a-b+ab)$ is $(a,b)$-representable. Indeed, $a(x'-1)+b(y'+a-
      1)=1-a-b+ab$.
    }
    \item{
      Let $n\geq ab-a-b+1$ and assume $n$ is $(a,b)$-representable. Let $x'$
      and $y'$ be as in \textbf{(b)}. Then there exist $x,y\in\N$ such that $ax+
      by=n$ and $0\leq x<b$.

      Note that $a(x+x')+b(y+y')=n+1$. If $y+y'\geq0$, then we are done.
      Otherwise, we wish to show $(x+x'-b,y+y'+a)\in\N^2$.

      Suppose $y+y'+a<0$ and recall from \textbf{(b)} that $y'+a\geq1$. We then
      have $y+y'<-a\leq y'-1$, or that $y<-1$. However, this implies
      \begin{align*}
          ab-a-b+1   &\leq n = ax+by < ax+b(-1)\\
          a(b-1)-b+1 &< ax-b\\
          a(b-1)     &< ax\\
          b          &\leq x
      \end{align*}
      But we assumed $0\leq x<b$, so this is a contradiction! Hence, $(y+y'+a)
      \in\N$.

      Next, note that $ab-a-b<n$ by assumption. Furthermore, $n+1=a(x+x')+b(y+
      y')\leq a(x+x')-b$, as $y+y'<0$. Since $n<n+1$, it follows that
      \begin{align*}
        ab-a-b   &< a(x+x')-b\\
        a(b-1)   &< a(x+x')\\
        b-1      &< x+x'\\
        b        &\leq x+x'
      \end{align*}
      Hence, $(x+x'-b)\in\N$ and $(x+x'-b,\,y+y'+a)$ is indeed a valid
      representation of $n+1$.
    }
    \qed
  \end{enumerate}

\subsection{} % 2.5
  Let $a,b\in\N$ be coprime with $ab\ne0$. Prove $|R_{a,b,c}|=\floor*{
  \frac{c}{ab}}$.

  Note that all solutions $(x,y)$ to the equation $ax+by=c$ are of the form
  $(cx'+kb,cy'-ka)$ for all $k\in\Z$ and for $(x',y')$ solving $ax'+by'=1$.
  Furthermore, if a solution $(u,v)\in R_{a,b,c}$, then $u,v\geq0$.

  Thus, $cx'+kb\geq0$ implies $k\geq-\frac{cx}{b}$ and $cy'-ka\geq0$ implies $k
  \leq\frac{cy'}{a}$. Hence,
  \begin{equation*}
    |R_{a,b,c}| = \floor*{\frac{cy'}{a}-\frac{-cx'}{b}}
                = \floor*{\frac{c}{ab}(y'b+x'a)}
                = \floor*{\frac{c}{ab}}
    \tag*{\qed}
  \end{equation*}

\subsection{} % 2.6
  Let $a,b\in\N$ be coprime with $ab\ne0$. Given $c\in\N\cap(0,ab)$ such that
  $a$ and $b$ do not divide c, prove $|R_{a,b,c}|+|R_{a,b,ab-c}|=1$.

  Define $\set{x}$ as $x-\floor*{x}$ and note $|R_{a,b,c}|=\frac{c}{ab}
  -\set{\frac{cy'}{a}}-\set{\frac{cx'}{b}}+1$ for $ax'+by'=1$. Then,
  \begin{align*}
    |R_{a,b,ab-c}|
      &= \frac{ab-c}{ab} -\left\{ \frac{(ab-c)y'}{a} \right\}
        -\left\{ \frac{(ab-c)x'}{b} \right\} +1\\
      &= 2-\frac{c}{ab} - \left\{ by'-\frac{cy'}{a} \right\}
        -\left\{ ax'-\frac{cx'}{b} \right\}\\
      &= 2-\frac{c}{ab} - \left(1 - \left\{ \frac{cy'}{a} \right\} \right)
        -\left(1-\left\{ \frac{cx'}{b} \right\}\right)\\
      &= -\frac{c}{ab}+\left\{\frac{cy'}{a}\right\}+\left\{\frac{cx'}{b}\right\}
  \end{align*}
  Thus,
  $
    |R_{a,b,c}|+|R_{a,b,ab-c}|
      = \left(
          \frac{c}{ab}-\left\{\frac{cy'}{a}\right\}-\left\{\frac{cx'}{b}\right\}+1
      \right) + \left(
        -\frac{c}{ab}+\left\{\frac{cy'}{a}\right\}+\left\{\frac{cx'}{b}\right\}
      \right)
      = 1
  $
  \qed

\end{document}
